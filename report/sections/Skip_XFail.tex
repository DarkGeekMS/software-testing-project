\subsection{Introduction}
Any software under test \emph{(SUT)} can have a set of infeasible test cases that can arise from dependencies issue or unimplemented functionalities. \textbf{pytest} offers some robust features to handle such cases using \emph{skip} and \emph{xfail} markers. In this section, we discuss both features in a bit of details. The scripts can be found under \texttt{test\_scripts/skip\_xfail\_test.py}, where the \emph{OTSU node} of the software under test is covered using both \emph{node} and \emph{edge} \textbf{coverage}. The script contains $5$ test cases :
\begin{enumerate}
    \item \textbf{test\_connection()} : node connection with outer sockets \emph{[SKIPPED]}.
    \item \textbf{test\_output()} : node output validity \emph{[PASSED]}.
    \item \textbf{test\_terminate()} : termination behaviour of node \emph{[XPASSED]}.
    \item \textbf{test\_wrong\_input()} : node behaviour with wrong inputs \emph{[FAILED]}.
    \item \textbf{test\_wrong\_port()} : node behaviour with wrong ports \emph{[XFAILED]}.
\end{enumerate}

\subsection{PyTest Skip Marker}

\subsubsection{Description}
\textbf{Skip marker} is usually used with tests that are expected to pass only under certain conditions. In other words, this feature is used by \textbf{pytest} to execute test cases, only if some conditions are met. This can be very useful for :
\begin{enumerate}
    \item Skipping test cases of unimplemented functionalities, which is crucial for \emph{test-driven development}.
    \item Skipping test cases on platform migration. This is used when migrating the software into another platform, where some implementations are not expected to run properly.
    \item Skipping test cases with unmet dependencies, where requirements for running code are unmet, due to incompatible versions or missing dependencies.
\end{enumerate}

\subsubsection{Usage}
\textbf{Skip marker} is mainly used through \emph{@pytest.mark.skip()} or \emph{@pytest.mark.skipif()} python decorators. It can be used to skip a certain test case and even a complete class or module. This feature is covered in the scripts by using both markers on \emph{test functions} or \emph{text classes}. \\

\emph{@pytest.mark.skip()} is used to skip a test case immediately without consideration :

\begin{python}
@pytest.mark.skip(reason="unimplemented functionality")
def test_case():
    ...
\end{python}

Meanwhile, \emph{@pytest.mark.skipif()} is used to skip a test case when a certain condition is met :

\begin{python}
import sys
@pytest.mark.skipif(sys.version_info < (3, 7), 
reason="requires python3.7 or higher")
def test_case():
    ...
\end{python}

\subsection{PyTest XFail Marker}

\subsubsection{Description}
\textbf{XFail marker} is usually used to indicate that a test is expected to fail. A common example is a test for a feature not yet implemented, or a bug not yet fixed. However, if the test passes, it will be reported as \emph{xpass}. Moreover, if the test fails, it will not be reported as a failure. The tester can mark a test as \emph{strict non-runnable xfail}, which will skip running the test. This can be useful to clean test report from any tests that are expected to fail, in order to only focus on meaningful failures.

\subsubsection{Usage}
\textbf{XFail marker} is mainly used through \emph{@pytest.mark.xfail()} python decorator. However, a set of parameters can be specified for extra functionalities :
\begin{enumerate}
    \item \textbf{condition} : defines a certain condition upon which the test is expected to fail.
    \item \textbf{reason} : defines the reason why test is expected to fail.
    \item \textbf{raises} : defines a set of exceptions that will be raised if test fails.
    \item \textbf{run} : a boolean that defines whether to run the test or totally ignore it.
    \item \textbf{strict} : a boolean that defines whether the test is a strict \emph{xfail}.
\end{enumerate}

\begin{python}
import sys
@pytest.mark.xfail(sys.platform == "win32",
reason="bug in a 3rd party library", raises=RuntimeError,
run=False, strict=True)
def test_case():
    ...
\end{python}