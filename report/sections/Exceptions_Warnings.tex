PyTest provides different methods to test some unexpected scenarios that throws exceptions and warnings. It helps to ensure that the exceptions and warnings are thrown correctly in the unexpected scenarios.
\subsection{Exceptions}

PyTest can assert about the expected exceptions using \textbf{pytest.raises()} in the context manager.
They are used on the SUT to test the failure when passing a non-valid input to the contours node or the otsu node and asserting that nothing will be returned from the contours node or the otsu node and \textbf{ZMQError} exception will be thrown in \textbf{test\_contour\_thread\_alive() and test\_otsu\_thread\_alive()}

\begin{python}
def test_contour_thread_alive(self):
    ......
    with pytest.raises(zmq.ZMQError):
        ......
        in_message = { 'binary' : 0.5} #passing not valid input
        ......
        ret_message = self.out_socket.recv_pyobj(flags = zmq.NOBLOCK)
\end{python}

It throws a \textbf{ZMQError} within the context manager.

\subsection{Warnings}

In a similar way to asserting that exceptions are thrown. In PyTest you can assert that a specific warning happens using \textbf{pytest.warns} inside the SUT \\
For example:
\begin{python}
def test_warning():
    with pytest.warns(UserWarning):
        warnings.warn("my warning", UserWarning)
\end{python}
In the the case of sending invalid inputs to the otsu and contours node and testing whether there is a UserWarning will show up, is evaluated in \textbf{test\_contour\_user\_warning() and test\_otsu\_user\_warning()}, For example:

\begin{python}
def test_contour_user_warning(self):
    .....
    with pytest.warns(UserWarning):
        .....
        in_message = { 'binary' : 0.5} #sending invalid input

\end{python}